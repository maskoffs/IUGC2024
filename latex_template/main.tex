% This is samplepaper.tex, a sample chapter demonstrating the
% LLNCS macro package for Springer Computer Science proceedings;
% Version 2.21 of 2022/01/12
%
\documentclass[runningheads]{llncs}
%
\usepackage[T1]{fontenc}
% T1 fonts will be used to generate the final print and online PDFs,
% so please use T1 fonts in your manuscript whenever possible.
% Other font encondings may result in incorrect characters.
%
\usepackage{graphicx}
\usepackage{hyperref} 
\usepackage{multirow}
\usepackage{amssymb}
\usepackage{booktabs}
\usepackage{silence}
\WarningFilter{latexfont}{Font}
% Used for displaying a sample figure. If possible, figure files should
% be included in EPS format.
%
% If you use the hyperref package, please uncomment the following two lines
% to display URLs in blue roman font according to Springer's eBook style:
\usepackage{color}
\renewcommand\UrlFont{\color{blue}\rmfamily}
% \urlstyle{rm}
%
\begin{document}
%
\title{Baseline Method at the Intrapartum Ultrasound Grand Challenge 2024}
%
%\titlerunning{.}
% If the paper title is too long for the running head, you can set
% an abbreviated paper title here
%
\author{Zihao Zhou}
%
%\authorrunning{Z. Zhou}
% First names are abbreviated in the running head.
% If there are more than two authors, 'et al.' is used.
%
\institute{College of Information Science and Technology, Jinan University, Guangzhou 510632, China
}
%
\maketitle              % typeset the header of the contribution
%

\begin{abstract}
....
\keywords{Intrapartum Ultrasound \and ...}
\end{abstract}


\section{Introduction}

This is citation \cite{ronneberger2015u}.

\section{Method}

\begin{figure*}[!b]
\centering
\includegraphics[width=\textwidth]{pictures/figure_1.png}
\caption{An overview of the baseline method network architecture.} 
\label{fig1}
\end{figure*}
\subsection{Network}

Fig. \ref{fig1} illustrates ... 



\begin{equation}
F^{\prime}= Conv\left( F_{\mathrm{in}} \right)
\end{equation}

\subsection{Loss Function}

\subsection{Training Details}

\subsection{Inference Details}


\section{Experiments and Results}
\subsection{Datasets}

\subsection{Metrics}


\subsection{Experiments}
Table \ref{tab1} presents the specific results.


\begin{table}[htbp]
  \renewcommand{\arraystretch}{1.25}
  \centering
  \caption{Experimental results on the validation and test sets.}
    \begin{tabular}{}

    \end{tabular}%
  \label{tab1}%
\end{table}%


\section{Discussion}
...


\section{Conclusion}
...

\bibliographystyle{splncs04}
\bibliography{ref}
%

\end{document}
